\documentclass[11pt]{article}
\usepackage{cite}
\usepackage[hidelinks]{hyperref}
\usepackage{graphicx}
\usepackage{subcaption}
\usepackage{xcolor}

\renewcommand{\labelenumii}{\theenumii}
\renewcommand{\theenumii}{\theenumi.\arabic{enumii}.}

\begin{document}
\pagenumbering{gobble}
\title{Machine learning on network analysis}
\author{Susmita Rai - 908928}
\date{}
\maketitle

\newpage
\pagenumbering{arabic}
\tableofcontents
\newpage

\setlength{\parskip}{1em}

\section{Project definition}
\subsection{Introduction}
With the rapid expansion of the Internet, it has become an essential part of our lives with over half the population connected \cite{wearesocial}. However, this results in an increasingly complex and fragile network. Many systems are left vulnerable, waiting to be exploited. By 2021, it is predicted that cost of cyber-attacks will reach \$6 trillion \cite{varonis}. The importance of a good and secure security system is far too crucial.

“Offensive cyber capabilities are developing more rapidly than our ability to deal with hostile incidents” \cite{globalrisks}. Attacks are becoming smarter, polymorphic viruses and obscured malwares are passing through current systems. Over a third of organizations believe that the threats they are facing cannot be blocked by their anti-virus \cite{varonis}. 
Due to our rapid growth, we have left many openings for an attack, one of them is through the network. Our need for constantly being connected is causing a major gap in security. In 2017, 8 different network attacks dominated the market \cite{network-attack-types}.

\begin{enumerate}
  \item Browser attacks - malicious users target vulnerable websites to infect, infecting new genuine users. 
  \item Brute force attack - attempting to guess your way through to the system. 
  \item Denial of service (DoS) or Distributed Denial of Service (DDoS) – flooding a service by creating many requests in order to slow or crash the system.
  \item Worm attacks – self propagating program that spreads through local system through exploitable vulnerabilities.
  \item Malware attacks – programs that can take many forms, however their purpose is always malicious.
  \item Web attacks – exploiting vulnerabilities found in the website such as SQL injection.
  \item Scan attacks – indirect attack to gain knowledge of any vulnerabilities that exist such as an open port. 
  \item Other attacks – attacks that were out of scope, such as physically attempting to steal device.
\end{enumerate}

Fortunately, methods such as Intrusion detection system (IDS) exist to deter most of these attacks. IDS constantly scan the network for any anomalous activity in the network. Some are even capable of stopping the attack completely rather than just alerting the user. 

However, IDS face many issues such as explaining what an anomaly is in the first place. Robustness and accuracy also come into question. How often does an IDS system report false negatives or how many different types of malwares can they detect? 

By using machine learning it is possible to overcome these problems because of its ability to learn patterns and understand different classifications. As malware evolve, the system can also be retrained to learn new patterns of attack. Because of this, I aim to create a robust system that is capable of detecting malwares using using machine learning.

\subsection{Project aims}
The project aims to create a system that can detect malware on a network by incorporating machine learning. The system should be able to classify a wide range of malwares accurately whilst having a low false positive rate. 

There are several different ways accomplish this. Below is list of steps I will take to approach this problem with several extensions that I would like to do, given time constraints and my own ability. 

\begin{enumerate}
  \item Develop a system capable of classifying different malwares. 
  
 This would be the foundation for all solutions to other problems. It would attempt to create an IDS system using machine learning techniques to detect anomalies, in this case malwares. Also, it aids to not only understand the dataset but the behaviour of malwares as well. 

  \item Create synthetic malware attack patterns

  This depends heavily on the results of step 1. If the system can detect malwares, then it can also learn the pattern of what makes a malware. This would allow creation of synthetic malware attack patterns. It is also possible to retrain the classifier in step 1 to be “smarter” against fake synthetic attacks which hopefully reduces the chances of a false positive, creating a smarter IDS system. 

  \begin{enumerate}
    \item{Replicate environment and create my own attacks}

   Another possibility I would like to do would be to create actual malware attacks, not just synthetic. Replicating the environment, the dataset was collected in, the classifier could be tested on a live network rather than just on a pre-collected dataset. 
  \end{enumerate}  

  \item Test robustness of available IDS with synthetic data.

 The final step would be to test how well IDS systems work. If synthetic malware data is realistic enough, it would be able to fool IDS systems into thinking an attack has happened. It is then easy to test an IDS system’s reaction to such attacks. If the system flags such an attack, it shows that there is room for improvement. 

  \begin{enumerate}
    \item{Create genetic adversarial networks to create even more realistic synthetic malware data.}

   With the usage of GANNs, more realistic synthetic malware data can be generated which in turn tests IDS systems. An almost infinite training loop could be created to train IDS systems to learn the correct patterns of malware which should make the system more robust. 
  \end{enumerate}  
\end{enumerate}

\newpage
\section{Related work}
\textcolor{red}{i dont fully understand whats supposed to be here}

The hunt for a high true positive with low false positive IDS system has been sought for a while. Variety of different algorithms have been implemented in order to find one true solution. 

Most attacks have a certain pattern that they follow resulting in two main approaches \cite{related-work-main-approaches}. Anomaly detection and signature-based detection. Given a normal day-to-day activity, if a deviation occurs then that is recorded as an anomaly. That is the rule that anomaly detectors follow. Whereas for signature-based detection, it’s based on the fact that an attack has a known pattern. If a known pattern is detected, there must have been an attack.

However, they both fall short. If a day-to-day activity is malicious, anomaly detectors will fail to flag it. Whereas for misuse detectors, the rise of polymorphic and obfuscated malwares is rendering it useless \cite{related-work-main-approaches}.

Nevertheless, many academics have implemented a range of different algorithms that attempts to detect malware. 

One paper introduces a novel method for signature-based intrusion detection \cite{related-work-signature-based}. Creating an updating database that stores most frequent signatures to detect. The results show that the rate of false positives lowered, and malware detection speed increased. However, the database requires the administrator to choose whether signature is dangerous or not. Furthermore, other paper criticizes signature-based detection for its lack of ability to detect innovative malwares \cite{related-work-criticise-signature}. 
 
By utilizing data mining techniques such as Bayesian networks, to select features of importance, they were able to detect various attacks such as DOS with 100\% accuracy. Their accuracy did falter for other attacks such as User2Root at 84\%. \cite{related-work-main-approaches}.

Further research has been done on applying algorithms that are capable of learning and understanding features on a massive scope. Genetic algorithms mixed with traditional algorithms resulted in lower false positives and false negatives \cite{related-work-criticise-signature}. 

Unfortunately, even with many different implements, current IDS systems present different results for the same situation \cite{related-work-advantages-and-disadvantages}. Judging the results can also be difficult. IDS systems that have high true positive and low false positive rate could lead to a false sense of security, especially if the attacks are over a large portion of traffic \cite{related-work-advantages-and-disadvantages}.

This shows that there’s progress still to make when implementing an IDS system that has high positive rate with low false positives. One that is also able to handle the constant face-paced changes of evolving malwares.

\newpage
\section{Project management}
\subsection{Methodology}
For this project, I have chosen to use Scrum agile methodology because of its iterative nature. Scrum framework follows one simple rule, that requirements can change, or otherwise known as requirements volatility \cite{methodology-requirement-volatile}. Due to its flexibility, Scrum is ideally designed for a team of ten or fewer \cite{methodology-ten-fewer} and since I will be working alone, some of its fundamentals will be altered

Usually a daily Scrum takes place at the start of the day where the development team discuss the events of yesterday, however since I will be working alone, no daily Scrum will take place. Instead, a meeting with all other undergraduates under the same supervisor will take place every fortnight. This way everyone can discuss the progress of their respective projects.

Whereas for sprint meetings, I will create a meeting with my supervisor every fortnight, which will be the length of one sprint. Instead of having a different meeting for planning, review and retrospective, it will all be merged into one. By doing this, I will still be able to receive some feedback and talk about issues that are/may cause problems.

I will also be responsible for creating a product and sprint backlog. Tasks will be broken down into separate categories, such as “todo” or “in-progress”. Each task will also be assigned a priority ranging from “must haves” to “would be nice”. I will be using GitHub project board \cite{methodology-github-projects} or some alternative that is similar in order to create the task board. Having an online task board allows easy access wherever I am. 

\begin{figure}[h!]
   \includegraphics[width=\linewidth, keepaspectratio]{"Pictures/machine-learning-life-cycle".png}
   \caption{Machine learning life cycle}
   \label{fig:ml-life-cycle}
\end{figure}

This methodology also ties in well with the structure for machine learning projects. Before a model can be created, data will have to be processed. In this step, data must be prepared and cleaned such that it can be fed into a model. This is then followed by model training and finally model testing and evaluation. Figure \ref{fig:ml-life-cycle} displays the life cycle of machine learning projects \cite{methodology-ml-life-cycle}. Depending on how the data is processed, the results will differ for the model. The algorithms used for the model also highly influences the outcome. This ends up being an iterative process since at each sprint, the goal would be to create the best model possible. 

\subsection{Risk analysis}

\begin{enumerate}
  \item{Time management}

  Due to my lack of productivity, unpredicted events or underestimation of the workload, the project may not be completed in time. However, because of my chosen software life cycle, I will be able to create a plan of tasks with their priority and effort levels recorded at the start of every sprint which should help guide the workflow. Also, the roadmap laid out by the Gantt chart contains a timeline which I will follow.

  \item{Hardware access}

  This project deals with big data which requires extensive hardware. Lack of hardware can hinder progress as methods will have to be optimized which will require more time. For example, trying to just load the data will have to be split into various batches as there won’t be enough RAM. Training models can also take significantly longer as a lot of processing power will be needed. Hopefully, this should not be an issue as the university has many different devices suited for this project. 

  \item{Lack of domain knowledge}

  This project deals with two separate issues, machine learning and security. For this project to succeed, it requires high domain knowledge in both fields which I may not have. However, by planning properly, tasks can be broken down into simpler components. Also, guidance is available from experts in the field.

  \begin{enumerate}
    \item{Dataset}

    Every machine learning project requires a dataset and since it is far too time consuming to produce my own \cite{methodology-general}, I will have to select an existing one. I may end up choosing a wrong dataset that is not fit for this project. Before I make a concrete decision on which dataset to choose from, I will ask for feedback from experts in the field.

    \item{Generalisable}

    A lot of IDS systems that have been implemented using machine learning techniques seem to not be generalizable to other networks or system \cite{methodology-general}. To avoid this, the model should not overfit the dataset, however if the model ends up underfitting, then the accuracy level will be poor. A balance between the two will be required. 
Creating my own hack and applying it to the model could also help learn patterns of malwares instead of learning what is the norm and any deviation from the norm is a malware. This could make it more generalizable.

    \item{Malware – features of importance}

    There are many algorithms that can understand what features are important and can find correlation between x and y easily. However, even if I created a list of features of importance, my lack of domain knowledge could make it harder to understand what the algorithm is trying to tell me. The results of the algorithm could be nonsense as well. There are many guides on known malwares, I will have to read up on them. If necessary, ask for expert advice as well.
  
    \item{Machine learning - applying right methods}

    Machine learning has a wide variety of different algorithms that could be implemented in this project. However, I may end up choosing the wrong ones. For example, when attempting to refine the model, I could end up implementing a strategy that does not make any sense. It could even lead my model to learn incorrect information.
Also, the first crucial step is to preprocess the data. Again, preprocessing it incorrectly could lead the model learning biased or incorrect information. For example, when the data contains a lot of NaNs, the model will not be able to handle this value. There are many ways of dealing with NaNs such as replacing it with a special value, using mean values or outright removing them. Whatever the method, it will affect the dataset which in turn will change the outcome of the model. A lot of background reading will be required and visualization of the dataset.
  \end{enumerate}
\end{enumerate}

\begin{figure}[h!]
   \includegraphics[width=\linewidth, keepaspectratio]{"Pictures/risk-analysis".png}
   \caption{Likelihood vs Impact}
   \label{fig:risk-analysis}
\end{figure}


\subsection{Gantt chart}
The Gantt chart is split into different sections however since the dates have not been finalised, a lot of end-dates have been estimated. 

The implementation section is split into two parts. There is an overall view of what I will be doing and it is also split into sprints to show specifically what I intend to do each sprint. The purple tasks are extensions that I would like to do if there is any time. There is also a block of sprints allocated incase I misjudged any timings.

\begin{figure}[!p]
    \centering
    \begin{subfigure}{\linewidth}
      \includegraphics[width=\linewidth]{Pictures/gantt-1".jpg}
    \end{subfigure}
    \begin{subfigure}{\linewidth}
      \includegraphics[width=\linewidth]{"Pictures/gantt-2".jpg}
    \end{subfigure}
    \caption{Gantt chart}
    \label{fig:image10}
\end{figure}




\newpage
\bibliography{ml-security-diss}
\bibliographystyle{unsrt}
\end{document}